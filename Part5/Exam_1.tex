\documentclass[12pt,fleqn]{report}
\usepackage{xcolor,framed}
\usepackage[francais]{babel}
\usepackage[T1]{fontenc}
\usepackage{color}
%\usepackage{amsnab}
\usepackage{mathrsfs}
\usepackage{pstricks}
\usepackage{pst-plot}
\usepackage{pst-node}
%\usepackage{textpath}
\usepackage{amssymb}
\usepackage{amsmath}
\usepackage{hhline}
\usepackage{longtable}
\usepackage{amscd}
\usepackage{theorem}
\usepackage{array}
\usepackage{delarray}
\usepackage{multicol}
\usepackage[dvips]{graphicx}


\paperheight=28cm
\paperwidth=22cm

  \setlength\textwidth{18cm}
  \hoffset=-1in
  \setlength\marginparsep{0cm}
  \setlength\marginparwidth{0cm}
  \setlength\marginparpush{0cm}
  \setlength\evensidemargin{2cm}
  \setlength\oddsidemargin{2cm}

  \setlength\topmargin{1.5cm}
  \setlength\headheight{1.5cm}
  \setlength\headsep{0cm}
  \voffset=-1in
  \setlength\textheight{23cm}

  \setlength{\parindent}{0mm}
  \setlength{\parskip}{1.5mm}




\newcommand{\abs}[1]{\left\lvert#1\right\rvert}
\newcommand{\C}{\mathbb{C}}
\newcommand{\Q}{\mathbb{Q}}
\renewcommand{\P}{\mathbb{P}}
\newcommand{\R}{\mathbb{R}}
\newcommand{\Z}{\mathbb{Z}}
\newcommand{\K}{\mathbb{K}}
\newcommand{\N}{\mathbb{N}}
\newcommand{\card}{Card}
\newcommand{\Dp}[2]{\dfrac{\partial {#1}}{\partial{#2}}}
\newcommand{\var}{\varepsilon}
\newcommand{\lda}{\lambda}
%\newcommand{\card}{card}
\newcommand{\bs}{\bigskip}
\newcommand{\ms}{\medskip}
\newcommand{\sk}{\smallskip}
\newcommand{\hr}{\hrulefill}
\newcommand{\Dint}{\displaystyle{\int}}
\newcommand{\dint}{\displaystyle{\int}}
\newcommand{\vsq}{\vspace*{-0.325cm}}
\renewcommand{\L}{\mathscr{L}}
\newcommand{\norm}[1]{\displaystyle{\left\|{#1}\right\|}}
\newcommand{\Inf}[1]{\displaystyle{\inf_{#1}}}
\newcommand{\Sup}[1]{\displaystyle{\sup_{#1}}}
\newcommand{\Max}[1]{\displaystyle{\max_{#1}}}
\newcommand{\Min}[1]{\displaystyle{\min_{#1}}}
\newcommand{\Sum}[2]{\displaystyle{\sum_{#1}^{#2}}}
\newcommand{\Int}[2]{\displaystyle{\int_{#1}^{#2}}}
\newcommand{\IInt}[1]{\displaystyle{\iint_{\!#1}}}
\newcommand{\IIInt}[1]{\displaystyle{\iiint_{\!#1}}}
\newcommand{\dd}{\textrm{\hspace{0.05cm}\textbf{d}}}
\newcommand{\ce}{\mbox{\textrm{\, c}}}
\newcommand{\wt}[1]{\widetilde{#1}}
\newcommand{\ov}[1]{\overrightarrow{#1\ }}
\setlength\parindent{0pt}
\newcommand{\Prod}[2]{\displaystyle{\prod_{#1}^{#2}}}
\newcommand{\absolue}[1]{\left| #1 \right|}
\newcommand{\Dlim}[2]{\displaystyle{\lim_{#1\rightarrow #2}}\,}
\newcommand{\di}{\displaystyle{\lim_{n\rightarrow +\infty}}\:}
\newcommand{\lz}{\displaystyle{\lim_{x\rightarrow 0}} \ }
\newcommand{\ORA}{\overrightarrow}
\newcommand{\fonc}[5]{#1 : \begin{cases}#2 \rightarrow #3 \\ #4 \mapsto #5
 \end{cases}}
\newcommand{\ppi}{ 3.1416 }
\newcommand{\pe}{ 2.718 }
\newcommand{\prd}{ 180 \ppi div mul } %% transforme des rd en degrés
\newcommand{\pdeg}{ \ppi mul 180 div } %% transforme des degrés en rd
\newcommand{\pExp}{ \pe exch exp } %% exponentielle
\newcommand{\pCos}{ \prd cos } %% cosinus (argument en radian)
\newcommand{\pSin}{ \prd sin }
\newtheorem{EX}{Exercice}
\newenvironment{exau}{\begin{EX} \normalfont}{\end{EX}}
\newtheorem{so}{Exercice}
\newenvironment{sol}{\begin{so} \normalfont}{\end{so}}
\newtheorem{sop}{Probl\`{e}me}
\newenvironment{solp}{\begin{sop} \normalfont}{\end{sop}}
\usepackage{fancyhdr}
\usepackage{tabularx}
\usepackage{lastpage}
\usepackage{color}
\usepackage{pstricks}
\usepackage{pstricks-pdf}
\usepackage[utf8]{inputenc}
\usepackage[T1]{fontenc}
%%%%
%%%%
%\newpsobject{showgrid}{psgrid}{subgriddiv=1,griddots=10,gridlabels=6pt}
\begin{document}



\begin{exau}
\ \hr \ \textbf{(5 Points)}

Soient 
$
A=\begin{pmatrix}
1&1&0\\ 0&1&1\\ 0&0&1
\end{pmatrix}
$ et $B = \begin{pmatrix} 0&1&0 \\ 0&0&1 \\ 0&0&0 \end{pmatrix} $
\begin{enumerate}
\item[${\rm a)}$] Calculer $(2I_3+A)$, ($2I_3+A - ~^tB)\times B$ (avec $~^tB$ la transposée de $B$), $B^2$ et $B^3$ . 
\item[${\rm b)}$] Donner une relation entre $I_3$, A et B. En déduire que A et B commutent (i.e., $AB=BA$).
\item[${\rm c)}$] Soit $n \in \matbb{N}$. Calculer $B^n$ selon les valeurs de n. En utilisant la formule de Newton et les éléments donnée à la fin de ce document, donnez une expression pour $A^n$.
\end{enumerate}
\end{exau}
%%%%%%%%%%%%%%%%%%
%%%%%%%%%%%%%%%%%%
%%%%%%%%%%%%%%%%%%
%%%%%%%%%%%%%%%%%% Exercice 2
%%%%%%%%%%%%%%%%%%
%%%%%%%%%%%%%%%%%%
%%%%%%%%%%%%%%%%%%



\begin{exau}
	\ \hr \ \textbf{(8 Points)}
	
Soient $\vec{n} =\begin{pmatrix} 1 \\ 2 \\3 \end{pmatrix}$ et $\vec{d} =\begin{pmatrix} 2 \\ 0 \\-2 \end{pmatrix}$.

Soient $A(2,0,-2)$, $B(2,3,-1)$ et  $C(0,2,-1)$. 
\begin{enumerate}
\item[${\rm a)}$] Calculer $\vec{n} \cdot \vec{d}$, $\vec{n} \wedge \vec{AC}$ et $||\vec{AB}||$
\item[${\rm b)}$] Calculer l'équation cartésienne du plan $(P)$ passant par les points A,B et C.
\item[${\rm c)}$] Calculer le système cartésien de la droite $\mathcal{D}$, de vecteur directeur $\vec{d}$ et telle que B appartienne à cette droite.
\item[${\rm d)}$] Ecrire le système permettant de trouver l'intersection de la droite et du plan, puis l'écrire sous forme matricielle $AX = Y$
\item[${\rm e)}$]
Soit $D(1,-2,5)$. On cherche un point $M(x,y,z)$ tel que la droite $(DM)$ soit perpendiculaire à la droite $\mathcal{D}$
\begin{enumerate}
\item[${\rm (i)}$] On suppose que $M \in \mathcal{D}$. Ecrire la condition de perpendicularité entre $(DM)$ et $\mathcal{D}$.
\item[${\rm (ii)}$] On prend un point $M(x,y,z)$ sur la droite $\mathcal{D}$. Ecrire les conditions d'appartenance de $M$ à $\mathcal{D}$, puis écrire et résoudre le système associé à l'appartenance de $M$ à $\mathcal{D}$ et la perpendicularité de $(DM)$ et $\mathcal{D}$. Conclure.
\end{enumerate}
\end{enumerate}
\end{exau}


%%%%%%%%%%%%%%%%%%
%%%%%%%%%%%%%%%%%%
%%%%%%%%%%%%%%%%%%
%%%%%%%%%%%%%%%%%% Exercice 3
%%%%%%%%%%%%%%%%%%
%%%%%%%%%%%%%%%%%%
%%%%%%%%%%%%%%%%%%

\begin{exau}
		\ \hr \ \textbf{(7 Points)}

Soit 
$$
A=\begin{pmatrix}
1&-2&6\\ 0&1&0\\ 1&0&1
\end{pmatrix}
$$
\begin{enumerate}
\item[${\rm a)}$] Montrer que $A$ est inversible. 
\item[${\rm b)}$] Trouver $A^{-1}$. 
\item[${\rm c)}$] Soit $\displaystyle{Y=\begin{pmatrix}-10\\ 3\\1 \end{pmatrix}}$. Résoudre le système $AX=Y$, avec ou sans calculs, et en justifiant.
\end{enumerate}

\end{exau}
\textbf{\large{BONUS}} \ \hr \ \textbf{(2 Points)}
\newline
Soient a et b deux réels non-nuls, on pose $A=\begin{pmatrix} a & b \\ 0&a \end{pmatrix}$. En posant une matrice $B=\begin{pmatrix} c &d \\ e&f \end{pmatrix}$, donner l'ensemble des matrices qui commutent avec $A$, c'est-à-dire telles que $AB=BA$.

\textbf{\large{Formule de Newton}} \hr \textbf{Q.1.c.}
\newline
 Soient 2 matrices $M$ et $N$ qui commutent, alors: $(M+N)^n = \sum_{k=0}^n \begin{pmatrix} n \\ k \end{pmatrix} M^k N^{n-k} = \begin{pmatrix} n \\ 0 \end{pmatrix}M^0N^n + \begin{pmatrix} n \\ 1 \end{pmatrix}M^1N^{n-1} + \begin{pmatrix} n \\ 2 \end{pmatrix}M^2N^{n-2} + \dots + \begin{pmatrix} n \\ n-1 \end{pmatrix}M^{n-1}N^{1} + \begin{pmatrix} n \\ n \end{pmatrix}M^{n}N^{0}$. 
 \newline
 On donne $\begin{pmatrix} n \\ 0 \end{pmatrix} = 1$, $\begin{pmatrix} n \\ 1 \end{pmatrix} = n$, $\begin{pmatrix} n \\ 2 \end{pmatrix} = \frac{n(n-1)}{2}$ et $B^0=I_3$. 
 \newline
 Pour information, $ \begin{pmatrix} n \\ k \end{pmatrix} = \frac{n!}{k!(n-k)!}$, avec $k \leq n$ des entiers et $n! = n \times (n-1) \times (n-2) \times \dots \times 2 \times 1$.

	\ \hr \ \textbf{Bonne Chance} \\
\textbf{EX1:=(0.5+1+0.5+0.5)+(0.5+0.5)+(0.75+0.75)\ \ \ \ EX2:=2+2+1+1+1+1\ \ \ \ EX3:=1.5+4+1.5}
%%%%%%%%%%%%%%%%%%
%%%%%%%%%%%%%%%%%%
%%%%%%%%%%%%%%%%%%
%%%%%%%%%%%%%%%%%% Exercice 6
%%%%%%%%%%%%%%%%%%
%%%%%%%%%%%%%%%%%%
%%%%%%%%%%%%%%%%%%

\end{document}