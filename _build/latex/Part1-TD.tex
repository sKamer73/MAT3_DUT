%% Generated by Sphinx.
\def\sphinxdocclass{jupyterBook}
\documentclass[letterpaper,10pt,english]{jupyterBook}
\ifdefined\pdfpxdimen
   \let\sphinxpxdimen\pdfpxdimen\else\newdimen\sphinxpxdimen
\fi \sphinxpxdimen=.75bp\relax
%% turn off hyperref patch of \index as sphinx.xdy xindy module takes care of
%% suitable \hyperpage mark-up, working around hyperref-xindy incompatibility
\PassOptionsToPackage{hyperindex=false}{hyperref}
%% memoir class requires extra handling
\makeatletter\@ifclassloaded{memoir}
{\ifdefined\memhyperindexfalse\memhyperindexfalse\fi}{}\makeatother

\PassOptionsToPackage{warn}{textcomp}

\catcode`^^^^00a0\active\protected\def^^^^00a0{\leavevmode\nobreak\ }
\usepackage{cmap}
\usepackage{fontspec}
\defaultfontfeatures[\rmfamily,\sffamily,\ttfamily]{}
\usepackage{amsmath,amssymb,amstext}
\usepackage{polyglossia}
\setmainlanguage{english}



\setmainfont{FreeSerif}[
  Extension      = .otf,
  UprightFont    = *,
  ItalicFont     = *Italic,
  BoldFont       = *Bold,
  BoldItalicFont = *BoldItalic
]
\setsansfont{FreeSans}[
  Extension      = .otf,
  UprightFont    = *,
  ItalicFont     = *Oblique,
  BoldFont       = *Bold,
  BoldItalicFont = *BoldOblique,
]
\setmonofont{FreeMono}[
  Extension      = .otf,
  UprightFont    = *,
  ItalicFont     = *Oblique,
  BoldFont       = *Bold,
  BoldItalicFont = *BoldOblique,
]


\usepackage[Bjarne]{fncychap}
\usepackage[,numfigreset=1,mathnumfig]{sphinx}

\fvset{fontsize=\small}
\usepackage{geometry}


% Include hyperref last.
\usepackage{hyperref}
% Fix anchor placement for figures with captions.
\usepackage{hypcap}% it must be loaded after hyperref.
% Set up styles of URL: it should be placed after hyperref.
\urlstyle{same}


\usepackage{sphinxmessages}



         \usepackage[Latin,Greek]{ucharclasses}
        \usepackage{unicode-math}
        % fixing title of the toc
        \addto\captionsenglish{\renewcommand{\contentsname}{Contents}}
        

\title{Travaux Dirigés}
\date{Sep 02, 2021}
\release{}
\author{Simon Kamerling}
\newcommand{\sphinxlogo}{\vbox{}}
\renewcommand{\releasename}{}
\makeindex
\begin{document}

\pagestyle{empty}
\sphinxmaketitle
\pagestyle{plain}
\sphinxtableofcontents
\pagestyle{normal}
\phantomsection\label{\detokenize{Part1/TD::doc}}


\begin{sphinxadmonition}{note}{Note:}
\sphinxAtStartPar
Dans tout le TD, on se place dans un repère orthonormé \(\mathcal{E} = (O,\vec{i},\vec{j},\vec{k})\).
\end{sphinxadmonition}

\sphinxAtStartPar
\sphinxstylestrong{Exercice 1.} On souhaite exprimer l’aire S d’un triangle à l’aide d’un produit vectoriel. On considère le triangle \(ABC\) et on note les coordonnées des points \(A(1;-2;7)\), \(B(2;2;1)\) et \(C(1;1;5)\).

\sphinxAtStartPar
Dans le plan \((ABC)\), les formules de trigonométrie donnent la relation suivante:
\begin{equation*}
\begin{split}
S=\frac{1}{2}\begin{Vmatrix}\vec{AB}\end{Vmatrix} \times \begin{Vmatrix}\vec{AC}\end{Vmatrix} \times \begin{vmatrix}sin\alpha\end{vmatrix}
\end{split}
\end{equation*}
\sphinxAtStartPar
où \(\alpha\) est l’angle orienté \((\vec{AB},\vec{AC})\).
\begin{itemize}
\item {} 
\sphinxAtStartPar
Exprimer l’aire \(S\) à l’aide d’un produit vectoriel.

\item {} 
\sphinxAtStartPar
Calculer l’aire \(S\).

\end{itemize}

\sphinxAtStartPar
\sphinxstylestrong{Exercice 2.} On considère le point \(A(-2;0;5)\) et le vecteur \(\vec{n}\begin{pmatrix}
2\\-6\\4
\end{pmatrix}\).

\sphinxAtStartPar
Déterminer une équation cartésienne du plan \(\mathcal{P}\) passant par \(A\) et de vecteur normal \(\vec{n}\).

\sphinxAtStartPar
\sphinxstylestrong{Exercice 3.} Soient les points \(A(1;-2;7)\), \(B(2;2;1)\) et \(C(1;1;5)\). En utilisant un produit vectoriel, déterminer une équation cartésienne du plan \((ABC)\).

\sphinxAtStartPar
\sphinxstylestrong{Exercice 4.} On considère la droite \((AB)\) avec \(A(1;2;-1)\) et \(B(0;1;3)\), ainsi que le plan \(\mathcal{P}\) d’équation : \(x+y+z-1=0\).
\begin{itemize}
\item {} 
\sphinxAtStartPar
Montrer que \((AB)\) et \(\mathcal{P}\) sont sécants.

\item {} 
\sphinxAtStartPar
Déterminer les coordonnées de leur point d’intersection \(I\). On commencera par déterminer une représentation paramétrique de \((AB)\).

\end{itemize}

\sphinxAtStartPar
\sphinxstylestrong{Exercice 5.} Montrer que les représentations paramétriques suivantes
définissent le même plan :
\begin{equation*}
\begin{split}
\left\{ \begin{array}{l}x=2+s+2t \\ y=2+2s+t \\ z=1-s-t \end{array} \right.
\quad \text{ et } \quad  \left\{ \begin{array}{l} x=1+3s'-t'\\ y=3+3s'+t'\\ z=1-2s' \end{array} \right.
\end{split}
\end{equation*}
\sphinxAtStartPar
\sphinxstylestrong{Exercice 6.}

\sphinxAtStartPar
  1. Déterminer la distance du point \(A\) au plan \((P)\)
\begin{itemize}
\item {} 
\sphinxAtStartPar
\(A(1,0,2)\) et \((P): 2x+y+z+4=0\).

\item {} 
\sphinxAtStartPar
\(A(3,2,1)\) et \((P): -x+5y-4z=5\).

\end{itemize}

\sphinxAtStartPar
  2. (** Un peu plus difficile) Calculer la distance du point \(A(1,2,3)\) à la droite
\((D):  \left\{ \begin{array}{l}
-2x+y-3z=1\\ x+z=1  \end{array} \right.\)







\renewcommand{\indexname}{Index}
\printindex
\end{document}