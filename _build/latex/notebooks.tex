%% Generated by Sphinx.
\def\sphinxdocclass{jupyterBook}
\documentclass[letterpaper,10pt,english]{jupyterBook}
\ifdefined\pdfpxdimen
   \let\sphinxpxdimen\pdfpxdimen\else\newdimen\sphinxpxdimen
\fi \sphinxpxdimen=.75bp\relax
%% turn off hyperref patch of \index as sphinx.xdy xindy module takes care of
%% suitable \hyperpage mark-up, working around hyperref-xindy incompatibility
\PassOptionsToPackage{hyperindex=false}{hyperref}
%% memoir class requires extra handling
\makeatletter\@ifclassloaded{memoir}
{\ifdefined\memhyperindexfalse\memhyperindexfalse\fi}{}\makeatother

\PassOptionsToPackage{warn}{textcomp}

\catcode`^^^^00a0\active\protected\def^^^^00a0{\leavevmode\nobreak\ }
\usepackage{cmap}
\usepackage{fontspec}
\defaultfontfeatures[\rmfamily,\sffamily,\ttfamily]{}
\usepackage{amsmath,amssymb,amstext}
\usepackage{polyglossia}
\setmainlanguage{english}



\setmainfont{FreeSerif}[
  Extension      = .otf,
  UprightFont    = *,
  ItalicFont     = *Italic,
  BoldFont       = *Bold,
  BoldItalicFont = *BoldItalic
]
\setsansfont{FreeSans}[
  Extension      = .otf,
  UprightFont    = *,
  ItalicFont     = *Oblique,
  BoldFont       = *Bold,
  BoldItalicFont = *BoldOblique,
]
\setmonofont{FreeMono}[
  Extension      = .otf,
  UprightFont    = *,
  ItalicFont     = *Oblique,
  BoldFont       = *Bold,
  BoldItalicFont = *BoldOblique,
]


\usepackage[Bjarne]{fncychap}
\usepackage[,numfigreset=1,mathnumfig]{sphinx}

\fvset{fontsize=\small}
\usepackage{geometry}


% Include hyperref last.
\usepackage{hyperref}
% Fix anchor placement for figures with captions.
\usepackage{hypcap}% it must be loaded after hyperref.
% Set up styles of URL: it should be placed after hyperref.
\urlstyle{same}


\usepackage{sphinxmessages}



         \usepackage[Latin,Greek]{ucharclasses}
        \usepackage{unicode-math}
        % fixing title of the toc
        \addto\captionsenglish{\renewcommand{\contentsname}{Contents}}
        

\title{Content with notebooks}
\date{Sep 02, 2021}
\release{}
\author{Simon Kamerling}
\newcommand{\sphinxlogo}{\vbox{}}
\renewcommand{\releasename}{}
\makeindex
\begin{document}

\pagestyle{empty}
\sphinxmaketitle
\pagestyle{plain}
\sphinxtableofcontents
\pagestyle{normal}
\phantomsection\label{\detokenize{notebooks::doc}}


\sphinxAtStartPar
You can also create content with Jupyter Notebooks. This means that you can include
code blocks and their outputs in your book.


\chapter{Markdown + notebooks}
\label{\detokenize{notebooks:markdown-notebooks}}
\sphinxAtStartPar
As it is markdown, you can embed images, HTML, etc into your posts!

\sphinxAtStartPar


\sphinxAtStartPar
You an also \(add_{math}\) and
\begin{equation*}
\begin{split}
math^{blocks}
\end{split}
\end{equation*}
\sphinxAtStartPar
or
\begin{equation*}
\begin{split}
\begin{aligned}
\mbox{mean} la_{tex} \\ \\
math blocks
\end{aligned}
\end{split}
\end{equation*}
\sphinxAtStartPar
But make sure you \$Escape \$your \$dollar signs \$you want to keep!


\chapter{MyST markdown}
\label{\detokenize{notebooks:myst-markdown}}
\sphinxAtStartPar
MyST markdown works in Jupyter Notebooks as well. For more information about MyST markdown, check
out \sphinxhref{https://jupyterbook.org/content/myst.html}{the MyST guide in Jupyter Book},
or see \sphinxhref{https://myst-parser.readthedocs.io/en/latest/}{the MyST markdown documentation}.


\chapter{Code blocks and outputs}
\label{\detokenize{notebooks:code-blocks-and-outputs}}
\sphinxAtStartPar
Jupyter Book will also embed your code blocks and output in your book.
For example, here’s some sample Matplotlib code:

\begin{sphinxVerbatim}[commandchars=\\\{\}]
\PYG{k+kn}{from} \PYG{n+nn}{matplotlib} \PYG{k+kn}{import} \PYG{n}{rcParams}\PYG{p}{,} \PYG{n}{cycler}
\PYG{k+kn}{import} \PYG{n+nn}{matplotlib}\PYG{n+nn}{.}\PYG{n+nn}{pyplot} \PYG{k}{as} \PYG{n+nn}{plt}
\PYG{k+kn}{import} \PYG{n+nn}{numpy} \PYG{k}{as} \PYG{n+nn}{np}
\PYG{n}{plt}\PYG{o}{.}\PYG{n}{ion}\PYG{p}{(}\PYG{p}{)}
\end{sphinxVerbatim}

\begin{sphinxVerbatim}[commandchars=\\\{\}]
\PYG{c+c1}{\PYGZsh{} Fixing random state for reproducibility}
\PYG{n}{np}\PYG{o}{.}\PYG{n}{random}\PYG{o}{.}\PYG{n}{seed}\PYG{p}{(}\PYG{l+m+mi}{19680801}\PYG{p}{)}

\PYG{n}{N} \PYG{o}{=} \PYG{l+m+mi}{10}
\PYG{n}{data} \PYG{o}{=} \PYG{p}{[}\PYG{n}{np}\PYG{o}{.}\PYG{n}{logspace}\PYG{p}{(}\PYG{l+m+mi}{0}\PYG{p}{,} \PYG{l+m+mi}{1}\PYG{p}{,} \PYG{l+m+mi}{100}\PYG{p}{)} \PYG{o}{+} \PYG{n}{np}\PYG{o}{.}\PYG{n}{random}\PYG{o}{.}\PYG{n}{randn}\PYG{p}{(}\PYG{l+m+mi}{100}\PYG{p}{)} \PYG{o}{+} \PYG{n}{ii} \PYG{k}{for} \PYG{n}{ii} \PYG{o+ow}{in} \PYG{n+nb}{range}\PYG{p}{(}\PYG{n}{N}\PYG{p}{)}\PYG{p}{]}
\PYG{n}{data} \PYG{o}{=} \PYG{n}{np}\PYG{o}{.}\PYG{n}{array}\PYG{p}{(}\PYG{n}{data}\PYG{p}{)}\PYG{o}{.}\PYG{n}{T}
\PYG{n}{cmap} \PYG{o}{=} \PYG{n}{plt}\PYG{o}{.}\PYG{n}{cm}\PYG{o}{.}\PYG{n}{coolwarm}
\PYG{n}{rcParams}\PYG{p}{[}\PYG{l+s+s1}{\PYGZsq{}}\PYG{l+s+s1}{axes.prop\PYGZus{}cycle}\PYG{l+s+s1}{\PYGZsq{}}\PYG{p}{]} \PYG{o}{=} \PYG{n}{cycler}\PYG{p}{(}\PYG{n}{color}\PYG{o}{=}\PYG{n}{cmap}\PYG{p}{(}\PYG{n}{np}\PYG{o}{.}\PYG{n}{linspace}\PYG{p}{(}\PYG{l+m+mi}{0}\PYG{p}{,} \PYG{l+m+mi}{1}\PYG{p}{,} \PYG{n}{N}\PYG{p}{)}\PYG{p}{)}\PYG{p}{)}


\PYG{k+kn}{from} \PYG{n+nn}{matplotlib}\PYG{n+nn}{.}\PYG{n+nn}{lines} \PYG{k+kn}{import} \PYG{n}{Line2D}
\PYG{n}{custom\PYGZus{}lines} \PYG{o}{=} \PYG{p}{[}\PYG{n}{Line2D}\PYG{p}{(}\PYG{p}{[}\PYG{l+m+mi}{0}\PYG{p}{]}\PYG{p}{,} \PYG{p}{[}\PYG{l+m+mi}{0}\PYG{p}{]}\PYG{p}{,} \PYG{n}{color}\PYG{o}{=}\PYG{n}{cmap}\PYG{p}{(}\PYG{l+m+mf}{0.}\PYG{p}{)}\PYG{p}{,} \PYG{n}{lw}\PYG{o}{=}\PYG{l+m+mi}{4}\PYG{p}{)}\PYG{p}{,}
                \PYG{n}{Line2D}\PYG{p}{(}\PYG{p}{[}\PYG{l+m+mi}{0}\PYG{p}{]}\PYG{p}{,} \PYG{p}{[}\PYG{l+m+mi}{0}\PYG{p}{]}\PYG{p}{,} \PYG{n}{color}\PYG{o}{=}\PYG{n}{cmap}\PYG{p}{(}\PYG{o}{.}\PYG{l+m+mi}{5}\PYG{p}{)}\PYG{p}{,} \PYG{n}{lw}\PYG{o}{=}\PYG{l+m+mi}{4}\PYG{p}{)}\PYG{p}{,}
                \PYG{n}{Line2D}\PYG{p}{(}\PYG{p}{[}\PYG{l+m+mi}{0}\PYG{p}{]}\PYG{p}{,} \PYG{p}{[}\PYG{l+m+mi}{0}\PYG{p}{]}\PYG{p}{,} \PYG{n}{color}\PYG{o}{=}\PYG{n}{cmap}\PYG{p}{(}\PYG{l+m+mf}{1.}\PYG{p}{)}\PYG{p}{,} \PYG{n}{lw}\PYG{o}{=}\PYG{l+m+mi}{4}\PYG{p}{)}\PYG{p}{]}

\PYG{n}{fig}\PYG{p}{,} \PYG{n}{ax} \PYG{o}{=} \PYG{n}{plt}\PYG{o}{.}\PYG{n}{subplots}\PYG{p}{(}\PYG{n}{figsize}\PYG{o}{=}\PYG{p}{(}\PYG{l+m+mi}{10}\PYG{p}{,} \PYG{l+m+mi}{5}\PYG{p}{)}\PYG{p}{)}
\PYG{n}{lines} \PYG{o}{=} \PYG{n}{ax}\PYG{o}{.}\PYG{n}{plot}\PYG{p}{(}\PYG{n}{data}\PYG{p}{)}
\PYG{n}{ax}\PYG{o}{.}\PYG{n}{legend}\PYG{p}{(}\PYG{n}{custom\PYGZus{}lines}\PYG{p}{,} \PYG{p}{[}\PYG{l+s+s1}{\PYGZsq{}}\PYG{l+s+s1}{Cold}\PYG{l+s+s1}{\PYGZsq{}}\PYG{p}{,} \PYG{l+s+s1}{\PYGZsq{}}\PYG{l+s+s1}{Medium}\PYG{l+s+s1}{\PYGZsq{}}\PYG{p}{,} \PYG{l+s+s1}{\PYGZsq{}}\PYG{l+s+s1}{Hot}\PYG{l+s+s1}{\PYGZsq{}}\PYG{p}{]}\PYG{p}{)}\PYG{p}{;}
\end{sphinxVerbatim}

\noindent\sphinxincludegraphics{{notebooks_2_0}.png}

\sphinxAtStartPar
There is a lot more that you can do with outputs (such as including interactive outputs)
with your book. For more information about this, see \sphinxhref{https://jupyterbook.org}{the Jupyter Book documentation}







\renewcommand{\indexname}{Index}
\printindex
\end{document}